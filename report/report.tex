\documentclass[a4paper]{article}
\usepackage[UTF8]{ctex}
\usepackage[left=3.17cm, right=3.17cm, top=2.54cm, bottom=2.54cm]{geometry}

\usepackage{fancyhdr}
\usepackage{graphicx, subfigure}
\usepackage{minted}
\setmonofont[]{Fira Code}
\usepackage[shortlabels]{enumitem}
\setenumerate{itemsep=0pt,partopsep=0pt,parsep=\parskip,topsep=0pt, itemindent=4em, leftmargin=0pt, listparindent=2em, label=(\arabic*)}
\setitemize{itemsep=0pt,partopsep=0pt,parsep=\parskip,topsep=5pt}
\setdescription{itemsep=0pt,partopsep=0pt,parsep=\parskip,topsep=5pt}

\usepackage[sort&compress]{gbt7714}
\usepackage{hyperref}
\hypersetup{unicode, colorlinks=true, linkcolor=black, urlcolor=black}
\usepackage{booktabs}
\usepackage{multirow}
\usepackage{color}
\usepackage{listings}

\definecolor{vgreen}{RGB}{104,180,104}
\definecolor{vblue}{RGB}{49,49,255}
\definecolor{vorange}{RGB}{255,143,102}

\lstset{ %
  language=verilog,                % the language of the code
  basicstyle=\footnotesize,           % the size of the fonts that are used for the code
  numbers=left,                   % where to put the line-numbers
  numberstyle=\tiny\color{gray},  % the style that is used for the line-numbers
  stepnumber=2,                   % the step between two line-numbers. If it's 1, each line 
                                  % will be numbered
  numbersep=5pt,                  % how far the line-numbers are from the code
  backgroundcolor=\color{white},      % choose the background color. You must add \usepackage{color}
  showspaces=false,               % show spaces adding particular underscores
  showstringspaces=false,         % underline spaces within strings
  showtabs=false,                 % show tabs within strings adding particular underscores
  frame=single,                   % adds a frame around the code
  rulecolor=\color{black},        % if not set, the frame-color may be changed on line-breaks within not-black text (e.g. commens (green here))
  tabsize=2,                      % sets default tabsize to 2 spaces
  captionpos=b,                   % sets the caption-position to bottom
  breaklines=true,                % sets automatic line breaking
  breakatwhitespace=false,        % sets if automatic breaks should only happen at whitespace
  title=\lstname,                   % show the filename of files included with \lstinputlisting;
                                  % also try caption instead of title
  keywordstyle=\color{blue},          % keyword style
  commentstyle=\color{dkgreen},       % comment style
  stringstyle=\color{mauve},         % string literal style
  escapeinside={\%*}{*)},            % if you want to add LaTeX within your code
  morekeywords={*,...}               % if you want to add more keywords to the set
}

\begin{document}

% \definecolor{bg}{rgb}{0.98,0.98,0.98}

\title{\textbf{流水线CPU处理器设计}}
\author{无03\quad 王与进 \quad2020010708}
\date{}

\maketitle

\tableofcontents

\newpage

\section{设计总览}

% \begin{figure}[ht]
%     \centering
%     \includegraphics[width=.8\textwidth]{../assets/design.png}
%     \caption{流水线总设计图}
%     \label{fig:总设计}
% \end{figure}

\section{实现细节}

\subsection{流水线阶段}

\subsubsection{流水线分段介绍——$\mathbf{IF}$}

\subsubsection{流水线分段介绍——$\mathbf{ID}$}

\subsubsection{流水线分段介绍——$\mathbf{EX}$}

\subsubsection{流水线分段介绍——$\mathbf{MEM}$}

\subsubsection{流水线分段介绍——$\mathbf{WB}$}

\subsection{冒险处理}

\subsection{外设}

\begin{lstlisting}[language=Verilog]      
    `timescale 1ns/1ps
    // Synchronous addition counter
    module sac10 (clk, rstn, count);
    input clk;
    input rstn;
    output reg [3:0] count;
    
    always @(posedge clk or posedge rstn) begin
        if(rstn)
            count <= 0;
        else if(count == 4'd9)
            count <= 0;
        else
            count <= count + 1;
    end
    endmodule //sac10
    
    
    module top (sys_clk, clk, an, ano, rstn, leds);
    input sys_clk, clk, rstn;
    input an;
    output ano;
    output [6:0] leds;
    wire [3:0] count;
    wire clk_o;
    
    assign ano = an;
    
    debounce debounce_instance(.clk(sys_clk),.key_i(clk),.key_o(clk_o));
    sac10 sac10_instance(.clk(clk_o),.rstn(rstn),.count(count));
    BCD7 bcd_instance(.din(count),.dout(leds));
    
    endmodule //top
\end{lstlisting}
    

\section{实验总结}
\begin{enumerate}
    \item \textbf{1}
    \item \textbf{2}
\end{enumerate}
    
\subsection{串口}

\newpage

\appendix

\end{document}